\documentclass[25pt]{sciposter}

\usepackage[T1]{fontenc}
\usepackage[utf8]{inputenc}

\usepackage{amsthm}

\usepackage[dvipsnames,usenames,svgnames,table]{xcolor} 
\usepackage{lipsum}
\usepackage{epsfig}
\usepackage{amsmath}
\usepackage{amssymb}
\usepackage[german]{babel}
\usepackage{geometry}
\usepackage{multicol}
\usepackage{graphicx}
\usepackage{tikz}
\usepackage{wrapfig}
\usepackage{gensymb}
\usepackage[utf8]{inputenc}
\usepackage{empheq}
\usepackage{mathtools}

\graphicspath{ {img/} }

\geometry{
 landscape,
 a1paper,
 left=5mm,
 right=50mm,
 top=5mm,
 bottom=50mm,
 }



%BEGIN LISTINGDEF



\usepackage{listings}
\usepackage{sourcecodepro}
\definecolor{listing-background}{rgb}{0.97,0.97,0.97}
\definecolor{listing-rule}{HTML}{B3B2B3}
\definecolor{listing-numbers}{HTML}{B3B2B3}
\definecolor{listing-text-color}{HTML}{000000}
\definecolor{listing-keyword}{HTML}{435489}
\definecolor{listing-identifier}{HTML}{435489}
\definecolor{listing-string}{HTML}{00999a}
\definecolor{listing-comment}{HTML}{8e8e8e}
\definecolor{listing-javadoc-comment}{HTML}{006CA9}

\lstdefinestyle{eisvogellistingstyle}{
	language=java,
	numbers=left,
	backgroundcolor=\color{listing-background},
	basicstyle=\color{listing-text-color}\small\ttfamily{}, % print whole listing small
	xleftmargin=0.8em, % 2.8 with line numbers
	breaklines=true,
	frame=single,
	framesep=0.6mm,
	rulecolor=\color{listing-rule},
	frameround=ffff,
	framexleftmargin=0.4em, % 2.4 with line numbers | 0.4 without them
	tabsize=4, %width of tabs
	numberstyle=\color{listing-numbers},
	aboveskip=1.0em,
	keywordstyle=\color{listing-keyword}\bfseries, % underlined bold black keywords
	classoffset=0,
	sensitive=true,
	identifierstyle=\color{listing-identifier}, % nothing happens
	commentstyle=\color{listing-comment}, % white comments
	morecomment=[s][\color{listing-javadoc-comment}]{/**}{*/},
	stringstyle=\color{listing-string}, % typewriter type for strings
	showstringspaces=false, % no special string spaces
	escapeinside={/*@}{@*/}, % for comments
	literate=
	{á}{{\'a}}1 {é}{{\'e}}1 {í}{{\'i}}1 {ó}{{\'o}}1 {ú}{{\'u}}1
	{Á}{{\'A}}1 {É}{{\'E}}1 {Í}{{\'I}}1 {Ó}{{\'O}}1 {Ú}{{\'U}}1
	{à}{{\`a}}1 {è}{{\'e}}1 {ì}{{\`i}}1 {ò}{{\`o}}1 {ù}{{\`u}}1
	{À}{{\`A}}1 {È}{{\'E}}1 {Ì}{{\`I}}1 {Ò}{{\`O}}1 {Ù}{{\`U}}1
	{ä}{{\"a}}1 {ë}{{\"e}}1 {ï}{{\"i}}1 {ö}{{\"o}}1 {ü}{{\"u}}1
	{Ä}{{\"A}}1 {Ë}{{\"E}}1 {Ï}{{\"I}}1 {Ö}{{\"O}}1 {Ü}{{\"U}}1
	{â}{{\^a}}1 {ê}{{\^e}}1 {î}{{\^i}}1 {ô}{{\^o}}1 {û}{{\^u}}1
	{Â}{{\^A}}1 {Ê}{{\^E}}1 {Î}{{\^I}}1 {Ô}{{\^O}}1 {Û}{{\^U}}1
	{œ}{{\oe}}1 {Œ}{{\OE}}1 {æ}{{\ae}}1 {Æ}{{\AE}}1 {ß}{{\ss}}1
	{ç}{{\c c}}1 {Ç}{{\c C}}1 {ø}{{\o}}1 {å}{{\r a}}1 {Å}{{\r A}}1
	{€}{{\EUR}}1 {£}{{\pounds}}1 {«}{{\guillemotleft}}1
	{»}{{\guillemotright}}1 {ñ}{{\~n}}1 {Ñ}{{\~N}}1 {¿}{{?`}}1
}
\lstset{style=eisvogellistingstyle}



%END LISTINGDEF


\newcommand*\widefbox[1]{\fbox{\hspace{2em}#1\hspace{2em}}}
\newcommand{\limm}{\lim\limits_{n \to \infty}}

\newlength\dlf  % Define a new measure, dlf
\newcommand\alignedbox[2]{
% Argument #1 = before & if there were no box (lhs)
% Argument #2 = after & if there were no box (rhs)
&  % Alignment sign of the line
{
\settowidth\dlf{$\displaystyle #1$}
    % The width of \dlf is the width of the lhs, with a displaystyle font
\addtolength\dlf{\fboxsep+\fboxrule}
    % Add to it the distance to the box, and the width of the line of the box
\hspace{-\dlf}
    % Move everything dlf units to the left, so that & #1 #2 is aligned under #1 & #2
\boxed{#1 #2}
    % Put a box around lhs and rhs
}
}
\usepackage{graphicx,url}

%BEGIN TITLE
\title{\huge{Analysis 1}}

\author{\large{David Zollikofer}}
%END TITLE

\usepackage{palatino}

% begin custom commands
\newcommand{\Q}{\mathbb{Q}}
\newcommand{\R}{\mathbb{R}}
\newcommand{\N}{\mathbb{N}}



\newtheorem{thm}{Thm}[section]



\usepackage[framemethod=TikZ]{mdframed}
\newenvironment{method}[1]{\begin{mdframed}[backgroundcolor=blue!10,innertopmargin=15pt, innerbottommargin=15pt, nobreak=true]
		\textbf{#1 }
	}
	{ 
	\end{mdframed}
}
\usepackage{todonotes}
\newcommand{\TODO}[1]{\todo[inline]{\Large TODO:  #1}}


\DeclarePairedDelimiter\abs{\left|}{\right|}%
\DeclarePairedDelimiter\norm{\lVert}{\rVert}%


\setlength\abovedisplayskip{0pt}


% end custom commands

\begin{document}

\fontfamily{ppl}\selectfont



\maketitle


\begin{multicols}{3}


\section{Einführung}
\begin{method}{Ordnungsvollständigkeit}
Unterscheidet $\R$ von $\Q$. Für $A,B$ nichtleere Teilmengen von $\R$ mit $$\forall a\in A \land \forall b \in B \implies a \leq b$$
Dann gibt es $c \in \R$ mit $\forall a \in A : a \leq c$ und $\forall b \in B : c \leq b$
\end{method}


\begin{method}{Archimedisches Prinzip}
Korollar von Ordnungsvollständigkeit. Sei $x\in \R$ mit $x>0$ und $y \in \R$, dann gibt es $n\in \N$ mit $y \leq nx$
\end{method}

\subsection*{Youngsche Ungleichung}
$\forall \epsilon > 0 , \forall x,y\in \R$ gilt: $2 |xy| \leq \epsilon x^2 + \frac{1}{\epsilon} y^2$. Beweis per Expansion von $\left( \sqrt{\epsilon} |x| - \frac{1}{\sqrt{\epsilon}} |y| \right) ^2 \geq 0$

\begin{method}{Existenz des Supremums}
Sei $A \subset \R$, $A \not = \varnothing$. Sei $A$ nach oben beschränkt, dann gibt es eine kleinste obere Schranke von $A$, $c = \sup A$, genannt Supremum von $A$. Um dies zu Beweisen nutzt man die Ordnungsvollständigkeit und definiert $B$ als die Menge der oberen Schranken.
\end{method}
\textbf{Beispiel (Sup \& Inf finden)} Sei $M = \{ \frac{|x|}{1 + |x|} : x \in \mathbb{R}\}$. \\ \textsc{Infinum:} Weil alle Elemente positiv sind ist von unten durch 0 beschränkt. Falls $x = 0$, folgt $\frac{|0|}{|0|+1} =0$ womit $\min{M} = \inf{M} = 0$. \\
\textsc{Supremum:} Für alle $x$ gilt $|x| + 1 > |x|$ somit $\frac{|x|}{|x|+1} < 1$. Somit gilt sicher $\sup M \leq 1$. \\
Nach Archimedes: $\forall \epsilon > 0,\exists n_0 : 1/n \leq \epsilon \ \forall n \geq n_0$
$$\frac{n}{n+1} = \frac{1}{1 + \frac{1}{n}} =\geq \frac{1}{1 + \epsilon} = 1 - \frac{\epsilon}{1 + \epsilon}$$
Wir erreichen $1$ sicher nie da sonst $1 = 0$, aber wir können beliebig nahe kommen. Somit $\sup M = 1$ aber $\neg \exists \max M$

\subsection*{Cauchy Schwarz}
$\forall x,y\in \R^n$ gilt $|\langle x,y\rangle| \leq ||x||\cdot||y||$.

\begin{method}{Komplexe Zahlen}\\
\includegraphics[scale=1.4]{complex.jpg}\\
\textsc{Division:} Es gilt $z^{-1} = \frac{\bar{z}}{||z||^2}$ wenn $z \not = 0$\\
\textsc{Polarform} Wenn \begin{align*}
z_1 &= r_1(\cos(\theta_1) + i\sin(\theta_1))\\
z_2 &= r_2(\cos(\theta_2) + i\sin(\theta_2))
\end{align*}
dann gilt: 
\begin{align*}
z_1 \cdot z_2 &= r_1 \cdot r_2 \left(\cos(\theta_1 + \theta_2) + i\sin(\theta_1 + \theta_2)\right)\\
\frac{z_1}{z_2} &= \frac{r_1}{r_2}\left(\cos(\theta_1 - \theta_2) + i\sin(\theta_1 - \theta_2)\right)
\end{align*}
Zudem folgt durch Induktion:
$$z^n = r^n (\cos(n\theta) + i \sin(n \theta))$$
\end{method}





% -------------------------- Folgen --------------------------

\section{Folgen}

\begin{method}{Definition Konvergenz}
Folge $a_n$ is konvergent, falls $\exists l \in \mathbb{R}$ so dass $\forall \epsilon > 0$ die Menge $\{n \in \mathbb{N}^+: a_n \not \in (l-\epsilon, l + \epsilon)\}$ endlich ist. \\
Äquivalent: $\forall \epsilon > 0 \exists N >  0$ s.d. $|a_n - l|< \epsilon$.
\end{method}

\textbf{Beispiel (Konvergenz mit Definition)}: 
Beweise dass $\limm \frac{3n^2 + 4}{2n^2 + 1} = \frac{3}{2}$. Sei $\epsilon > 0$. Dann muss gelten $|a_n - a | < \epsilon$. Somit $\left| \frac{3n^2 + 4}{2n^2 + 1} -  \frac{3}{2}\right| = \left| \frac{6n^2 + 8 - 6n^2 -3}{4n^2 + 2} \right| = \frac{5}{4n^2 + 2} < \epsilon$. Gibt $4n^2 + 2 > \frac{5}{\epsilon}$, äquiv zu $n > \sqrt{\frac{5}{4 \epsilon} - \frac{1}{2}}$. Wir definieren $N = \lfloor \sqrt{\frac{5}{4 \epsilon} - \frac{1}{2}} \rfloor$. Nun gilt $\forall n \geq N : |a_n - a | < \epsilon$.




\begin{method}{Rechnen mit Grenzwerten} $(a_n)_n$ und $(b_n)_n$ konvergent mit GW $a,b$. Dann gilt:
	\begin{itemize}
		\item $\lim \limits_{n \to \infty} (a_n + b_n) = a + b$	
		\item $\lim \limits_{n \to \infty} (a_n \cdot b_n) = a \cdot b$
		\item  Falls $b_n, b \not = 0$, so gilt $\lim \limits_{n \to \infty} \frac{a_n}{b_n} = \frac{a}{b}$
		\item Falls $a_n \leq b_n$ $\forall n \in \mathbb{N}$, so gilt $a \leq b$
	\end{itemize}
\end{method}

\textbf{Beispiel (Rechnen mit Grenzwerten)}: 
\begin{itemize}
	\item $\limm  \frac{(\frac{1}{n} + n^2 )^3}{1 + n^6} = \limm \frac{n^6}{n^6} \frac{(\frac{1}{n^3} + 1 )^3}{\frac{1}{n^6} + 1} = \limm \frac{(\frac{1}{n^3} + 1 )^3}{\frac{1}{n^6} + 1} = 1$
	
	\item $\limm \sqrt{n} (\sqrt{n + 2} - \sqrt{n}) \stackrel{\text{Wurzeltrick}}{=} \limm \frac{2 \sqrt{2}}{\sqrt{n+2} + \sqrt{n}} = \limm  \frac{2}{\sqrt{1 + \frac{2}{n}} + 1} = 1$
\end{itemize}

\begin{method}{Satz von Weierstrass}
	Wenn $a_n$ nach oben (nach unten) konvergent ist und monoton wachsend (fallend), dann ist $a_n$ konvergent.
\end{method}

\textbf{Beispiel (Induktive Folge mit Weierstrass)}: 
Beachte $a_0 = 0$, $a_{n+1} = \left(\frac{a_n}{2}\right)^2 + 1$. \textsc{Monotonie:} Anstatt $a_{n+1} > a_{n}$ zeigen wir $a_{n+1} - a_{n} > 0$:
$$a_{n+1} - a_{n} = \frac{a_n ^2}{4} + 1 - a_n = \frac{a_n ^2 -4a_n + 4}{4} = \frac{(2-a_n)^2}{4} \geq 0$$
\textsc{Beschränktheit:} ($\leq 2$) per Induktion. Für $n=0$, $a_0 = 0 \leq 2 \checkmark$. Schritt: $a_{n+1} = \frac{a_n ^2}{4} + 1 \stackrel{\text{I.H.}}{\leq} \frac{2^2}{4} +1 = 2 \checkmark$. Nach Weierstrass konvergent. mit $a = \frac{a^2}{4} + 1 \implies a = 2$ als GW.

\begin{method}{Sandwich Theorem}
Seien $b_n \leq a_n \leq c_n$. Falls $b_n$ und $c_n$ konvergent sind mit $\lim \limits_{n \to \infty } b_n = \lim \limits_{n \to \infty } c_n = L$, so ist $a_n$ auch konv. mit $\lim \limits_{n \to \infty } a_n = L$
\end{method}
\textbf{Beweis (Sandwich Theorem)}: 
Da $b_n$, $c_n$ konvergent mit GW $L$:
$$\forall \epsilon > 0 \exists N_1 : \forall n \geq N_1 : L-\epsilon < b_n + \epsilon$$
Analog  $ \exists N_2$, setze $N = \max\{N_1, N_2\}$, dann gilt $\forall \epsilon > 0 \exists N$:
$$L-\epsilon < b_n \leq a_n \leq c_n < L + \epsilon \implies L - \epsilon < a_n < L + \epsilon$$


\textbf{Beispiel (Sandwich)}
$\limm \frac{n + \cos(n)}{n^2 -1}$. Wenn es absolut konv. dann auch absolut. Sicher gilt somit : $0 \leq \left|\frac{n + \cos(n)}{n^2 -1}\right|$ Nun schätzen wir nach oben ab: $\left|\frac{n + \cos(n)}{n^2 -1}\right| \leq \frac{n+1}{n^2 - 1} = \frac{n+1}{(n+1)(n-1)} = \frac{1}{n-1}$. Es gilt $\limm \frac{1}{n-1} = 0$, somit gilt $\limm \left|\frac{n + \cos(n)}{n^2 -1}\right| = 0$. (da Sandwich)

\begin{method}{Cauchy Kriterium}
	Die Folge $(a_n)_n$ ist genau dann konvergent, falls $\forall  \epsilon > 0 \ \exists N \geq 1 \text{ so dass } |a_n - a_m| < \epsilon \quad \forall n,m \geq N$
\end{method}

\textbf{Beispiel (Cauchy Kriterium)}
Sei $a_n = \frac{n-1}{2n}$ eine Folge. Wir versuchen Konvergenz mittels Cauchy zu zeigen:\\
$|a_n - a_m| = |\frac{n-1}{2n} - \frac{m-1}{2m}| = |\frac{nm-m-nm+n}{2nm}| = |\frac{n-m}{2nm}| = \frac{1}{2n} - \frac{1}{2m} \stackrel{m\geq n}{\leq} \frac{1}{2n} < \epsilon$. Für alle $n \geq N = \lceil \frac{1}{2\epsilon} \rceil$ haben wir $|a_n - a_m| < \epsilon$. Nach Cauchy ist $a_n$ somit konvergent.

\TODO{SATZ VON BOLZANO WEIERSTRASS}


% -------------------------- Reihen --------------------------

\section{Reihen}

\begin{method}{Konvergenz mit Definition} Eine Reihe konvergiert wenn der Grenzwert der Partialsummen existiert:
	$$\sum_{n=0}^{\infty} a_n  = \lim\limits_{N\to\infty} S_n = \lim\limits_{N\to\infty} \sum_{n=0}^{\infty} a_n$$
\end{method}

\textbf{Beispiel (Konvergenz mit Definition)}
\begin{itemize}
	\item $\sum_{n = 0}^{\infty} \frac{1}{n(n+1)} = \frac{1}{n} - \frac{1}{n+1}$. Es gilt somit $S_n = 1 - \frac{1}{2} + \frac{1}{2} - \frac{1}{3} \ldots - \frac{1}{n+1}$ Somit $S_n = 1-\frac{1}{n+1}$. Es gilt $\limm S_n = \limm 1 - \frac{1}{n+1} = 1$, somit ist die Reihe konvergent mit Grenzwert 1.
	\item $\sum_{n = 0}^{\infty} \frac{1}{\sqrt{n} + \sqrt{n+1}} $ Zuerst nutzen wir den Wurzeltrick: $\frac{1}{\sqrt{n} + \sqrt{n+1}} = \sqrt{n+1} - \sqrt{n}$. Wir bemerken $S_n = 1 - 0 + \sqrt{2} - 1 \ldots + \sqrt{n+1} - \sqrt{n} = \sqrt{n+1}$. Da aber $\limm S_n =  \limm \sqrt{n+1} = \infty$ divergiert die Reihe.
\end{itemize}



\begin{method}{Konvergenz mit $\lim\limits_{n \to \infty } a_n = 0$}
$$\sum_{n=0}^{\infty} a_0 \text{ konvergiert } \implies \lim\limits_{n \to \infty } a_n = 0 $$
\end{method}

\textbf{Beweis (Nullfolge)} Wenn $\sum_{n = 0}^{\infty} a_n$ konvergent ist, dann ist $\limm a_n = 0$.
Es gilt $a_N = S_N - S_{N-1}$. Somit $\limm a_n = \limm (S_n - S_{n-1}) = \limm S_n - \limm S_{n-1} = S-S = 0$

\begin{method}{Majorantenkriterium}
	Fall $a_n \geq b_n \forall n \geq n_0$ für ein $n_0$. Dann gilt $\sum_n a_n \text{ konvergiert } \implies \sum_{n} b_n$ konvergiert.
\end{method}

\begin{method}{Minorantenkriterium}
	Fall $a_n \geq b_n \forall n \geq n_0$ für ein $n_0$. Dann gilt $\sum_n b_n \text{ divergiert } \implies \sum_{n} a_n$ divergiert.
\end{method}




\begin{method}{Quotientenkriterium}
	Sei $a_n \not = 0$, dann gilt:
	\begin{itemize}
		\item $\lim\limits_{n \to \infty} |\frac{a_{n+1}}{a_n}|> 1 \implies \sum_{n} a_n$ divergiert. 
		\item $\lim\limits_{n \to \infty} |\frac{a_{n+1}}{a_n}|< 1 \implies \sum_{n} a_n$ konvergiert. 
		\item $\lim\limits_{n \to \infty} |\frac{a_{n+1}}{a_n}|=  1 \implies $ Kriterium versagt.
	\end{itemize}
\end{method}

\textbf{Beweis (Quotientenkriterium)} Angenommen $c_n = \sup\{ \frac{|a_{k+1}|}{|a_{k}|} : k \geq n \}$. Dann wenn beschränkt: $\limsup_{n\to \infty} \frac{|a_{n+1}|}{|a_{n}|} = \limm c_n$. Angenommen wir haben $q$ so dass $\limm c_n < q < 1$. Sei $N \geq 1$ so dass $c_N \leq q < 1$ woraus $\frac{|a_{k+1}|}{|a_k|} \leq q$ $\implies |a_{k+1}| \leq q |a_k|$. Für $j\geq 1$ folgt: $|a_{N+j}|\leq q|a_{N+j-1}| \leq \ldots \leq q^j |a_N| = q^{N+j} \frac{|a_N|}{q^N}$. Somit gilt: $|a_n| \leq q^n \frac{|a_N|}{q^N}$ und wir haben eine Majorante.

\textbf{Beispiel (Quotientenkriterium)}

\begin{itemize}
	\item $\sum_{n = 1}^\infty \frac{4^n}{n!}$. $\limm \frac{|a_{n+1}|}{|a_n|} = \limm \frac{4^{n+1} n!}{4^n (n+1)!} = \limm \frac{4}{n+1} = 0 < 1$. Konvergent $\checkmark$
	
	\item $\sum_{n = 1}^\infty \frac{n^n}{(2n)!!}$.  $\limm \frac{|a_{n+1}|}{|a_n|}  = \frac{(n+1)^{n+1}}{(2n+2)!!} \frac{(2n)!!}{n^n} = \limm \frac{(n+1)^{n+1}}{n^n} \frac{(2n)!!}{(2n+2)(2n)!!} = \limm \frac{1}{2} \left( \frac{n+1}{n} \right)^n = \frac{e}{2}>1$. Somit divergent.
\end{itemize}





\begin{method}{Wurzelkriterium}
	\begin{itemize}
		\item $\lim\limits_{n \to \infty} \sqrt[n]{|a_n|}>1 \implies \sum_{n} a_n $ divergiert
		\item $\lim\limits_{n \to \infty} \sqrt[n]{|a_n|}<1 \implies \sum_{n} a_n $ konvergiert
		\item $\lim\limits_{n \to \infty} \sqrt[n]{|a_n|}=1 \implies $ Kriterium versagt
	\end{itemize}
\end{method}

\textbf{Beispiel (Wurzelkriterium)}
\begin{itemize}
	\item $\sum_{n=1}^{\infty} \left( n^{1/n} -1 \right)^n $. Es gilt $\limm \sqrt[n]{|a_n|} = \limm (n^{1/n}-1) = \limm e^{\frac{1}{n}\log n} -1 = e^0-1 = 0$ Somit konvergent.
	
	\item $\sum_{n=1}^{\infty} \left(\frac{n}{2n+1}\right) ^{n-5}$. Es gilt $\limm \sqrt[n]{|a_n|} = \limm \left( \frac{n}{2n+1} \right)^{1-\frac{5}{n}} = \limm\frac{n}{2n+1} = \frac{1}{2}$. Somit konvergent.
\end{itemize}


\textbf{Beweis (Wurzelkriterium)}
Sei $c = \sup \{\sqrt[k]{|a_k|}\}$ und $ < q< 1$ mit $\limm c_n = \limsup_{n\to \infty} \sqrt[k]{|a_k|} < q <1$. Dann $\exists N \geq 1$ mit $ c_N = \sup\{ \sqrt[k]{|a_k|}: k\geq N \} \leq q$ was uns $|a_n| \leq q^n$ gibt. Somit ist eine geometrische Reihe eine Majorante.


\begin{method}{Integralkriterium} Wenn $\sum_{n=p}^{\infty} a_n$ sowohl $a_n \geq 0$ und $a_{n+1} \leq a_n$ (monot. fall.), dann gilt:
$$\sum_{n=p}^{\infty} a_n \text{ konv. } \iff \int_{p}^{\infty} a(x) dx \text{ konv.}$$
\end{method}
\TODO{Frage ob legal}


\begin{method}{Leibnitzkriterium} Eine Reihe $\sum_n (-1)^n a_n$ konvergiert falls $a_n \geq 0$ , $\lim\limits_{n \to \infty } a_n = 0$ und $a_n$ monoton fallend ist.
\end{method}

\textbf{Beispiel (Leibnitz)}: $\sum_{n = 1}^{\infty} \frac{\cos(2\pi n)}{\cos(n \pi)} \frac{\log(n)}{n^2}$. Wir stellen fest dass $\sum_{n = 1}^{\infty} \frac{\cos(2\pi n)}{\cos(n \pi)} \frac{\log(n)}{n^2} = \sum_{n = 1}^{\infty} (-1)^n \frac{\log(n)}{n^2}$. Zudem gilt $\frac{\log(n)}{n^2} \geq 0$, sowie $\lim\limits_{n \to \infty} \frac{\log(n)}{n^2} = 0$. Zudem ist $\frac{\log(n)}{n^2}$ monot. fallend für $n \geq 2$, da: 
$$ \frac{d}{dx} \frac{\log(x)}{x^2} = -2 \frac{\log(x)}{x^3} + \frac{1}{x} = \frac{1- 2\log(x)}{x^3} \leq 0\text{ } \forall x \geq \sqrt{e} = 1.6..$$





\begin{method}{Absolute Konvergent}
	Eine Reihe $\sum_{n} a_n$ ist absolut konvergent falls $\sum_{n} |a_n|$ konvergiert.
\end{method}

\textbf{Beispiel (Absolute Konvergenz)} Ang. $\sum_{n} a_n$ konvergiert absolut. Betrachte $\sum_{n=0}^{\infty} (\sqrt{1 + a_n} -1)$ und bemerke dass es auch abs. konv.
$$\left|\sqrt{1 + a_n} -1\right| = \left| (\sqrt{1 + a_n} -1) \frac{\sqrt{1 + a_n} +1}{\sqrt{1 + a_n} +1} \right| = \left| \frac{a_n}{\sqrt{1 + a_n} + 1} \right| \leq |a_n|$$

\TODO{Sätze aus dem Skript zur absoluten Konvergenz}


\TODO{Satz Weierstrass / Bolzano Weierstrass}

\newpage

\end{multicols}
\end{document}